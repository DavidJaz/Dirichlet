
A polynomial functor $P : \Set \to \Set$ is a sum of representables
$$P(X) := \sum_{b \in B} X^{E_b}$$
and therefore depends on a $B$-indexed family of sets $E_b$. This data is known
in the computer science literature as a ``container'', but as such an indexed
family of sets can be represented by as a function
\[
  \begin{tikzcd}
    E \arrow[d, "\pi"] \\
    B
  \end{tikzcd}
\]
we will refer to it as a \emph{bundle}. We think of a bundle as a categorified
Young diagram:

\[
[image]
\]

$B$ is the set of rows and $E_b$ is the set of boxes in row $b$ (possibly empty,
for generalities sake). We can see the polynomial functor
$$P(X) := \sum_{b \in B} X^{E_b}$$
as sending a set $X$ to the set of ways to label a row of the correponding Young
diagram with elements of $X$.

Remarkably, all natural transformations between polynomial functors can be
represented in terms of their associated bundles. A natural transformation $P
\to P'$ corresponds to a pair of functions
\[
\left( f : B \to B',\, f^{\sharp} : \dprod{b : B} E'_{fb} \to E_b \right)
\]
(using dependent type syntax) acting covariantly on rows and contravariantly on
boxes in each row. In terms of bundles, this is a diagram of the following sort:

\[
  \begin{tikzcd}
    E \arrow[d, "\pi"'] \arrow[r, leftarrow, "f_{\sharp}"] & \bullet \arrow[d] \arrow[r]
    \arrow[dr, phantom, "\ulcorner" very near start] & E' \arrow[d, "\pi'"] \\
    B \arrow[r, equals] & B \arrow[r, "f"'] & B'
  \end{tikzcd}
\]
We refer to these special spans as \emph{contravariant} morphisms of bundles.

\begin{thm}[cite]
The category of polynomial functors and natural transformations is equivalent to
the category of bundles and contravariant morphisms.
\end{thm}

This begs the question: what, then, are we to make of the more obvious,
\emph{covariant} morphisms of bundles
\[
  \begin{tikzcd}
E \arrow[r, "\term{tot}(f_{\sharp})"] \arrow[d, "\pi"'] & E' \arrow[d, "\pi'"] \\
B \arrow[r, "f"']                                       & B'                  
\end{tikzcd}
\]
which, in terms of Young diagrams, are covariant on both the rows and the boxes
in each row.

It turns out that these covariant maps of bundles correspond to natural
transformations between the appropriate sums of \emph{contravariant}
representables:
$$D(X) := \sum_{b \in B} E_b^X$$
In terms of Young diagrams, such a functor takes a set $X$ to the set of ways to
place the elements of $X$ in the boxes of a row.

Polynomial functors get their name because when $B$ and $E$ are finite, one may
organize the exponents $E_b$ by cardinality, yielding a function
$\term{card}E_{(-)} : B \to \Nb$. Letting $B_n := (\term{card}E_{(-)})\inv(n)$,
we find that the cardinality
$$|P(X)| = \sum_{n \in \Nb} |B_n||X|^n$$
is a polynomial in the cardinality of $X$. Similarly,
$$|D(X)| = \sum_{n \in \Nb} |B_n|n^{|X|}$$
resembles a Dirichlet series in the cardinality of $X$. Accoringly, we call such
sums of contravariant representables \emph{Dirichlet functors}.

We will show in this paper that Dirichlet functors are, quite robustly, the
contravariant analogue of polynomial functors. In particular, the many ways to
say that a functor is polynomial have contravariant analogues.
\begin{thm}[cite]\label{thm:polynomial.set.characterization}
  Let $P : \Set \to \Set$ be a functor. Then the following are equivalent.
  \begin{enumerate}
  \item $P$ is polynomial.
  \item $P$ is the sum of representables.
  \item There is a bundle $\pi : E \to B$ together with a natural isomorphism
    $$P(X) \cong \sum_{b \in B} X^{E_b}.$$
    Or, equivalently, a natural isomorphism of $P$ with the composite
    $$\Set \xto{E^{\ast}} \Set_{/E} \xto{\pi_!} \Set_{/B} \xto{B_\ast} \Set.$$
  \item $P$ is accessible and preserves connected limits.
  \end{enumerate}
\end{thm}

Analogously, we will prove the following theorem.
\begin{thm}\label{thm:dirichlet.set.characterization}
Let $D : \Set\op \to \Set$ be a contravariant functor. Then the following are
equivalent.
\begin{enumerate}
\item $D$ is Dirichlet.
\item $D$ is the sum of contravariant representables.
\item There is a bundle $\pi : E \to B$ together with a natural isomorphism
  $$D(X) \cong \sum_{b \in B} E_b^X.$$
  Or, equivalently, a natural isomorphism of $D$ with the composite
  $$\Set\op \xto{(B^{\ast})\op} (\Set_{/B})\op \xto{[-, \pi]} \Set_{/B}
  \xto{B_{\ast}} \Set.$$
\item $D$ preserves connected limits.
\end{enumerate}
\end{thm}

Note that we no longer need to assume accessiblity. This is a general feature of
the theory of Dirichlet functors; it is a bit ``smaller'' and more maneageable
than that of polynomials. In particular, a Dirichlet functor is determined by
its action on the terminal morphism $!_0 : 0 \to \ast$ of the empty set. As a corollary, Dirichlet functors form a topos.

\begin{thm}\label{thm:dirichlet.set.equivalence}
The functor $\Set^{\downarrow} \to \type{Fun}(\Set\op, \Set)$ sending $\pi : E
\to B$ to the induced Dirichlet functor $X \mapsto \sum_{b \in B} E_b^X$ is
fully faithful, and so gives an equivalence
$$\Set^{\downarrow} \simeq \Dir$$
between bundles and Dirichlet functors, with inverse given by evalutation at
$!_0 : 0 \to \ast$.
\end{thm}


Now, object-wise, a Dirichlet functor and a polynomial functor are determined by
the same data --- a bundle $\pi : E \to B$ of sets. Accordingly, one would
expect a transformation
$$X^N \mapsto N^X$$
turning polynomial functors into Dirichlet functors, and vice versa. But the
natural transformations between each sort of functor induce different morphisms
between the bundles; natural tranformations between polynomial functors induce
contravariant bundle morphisms, while natural transformations between Dirichlet
functors induce covariant bundle morphisms. However, if we restrict to those
morphisms of bundles which are \emph{isovariant} on the fibers --- that is, the
pullback diagrams of the form
\[
  \begin{tikzcd}
E \arrow[r, "\term{tot}(f_{\sharp})"] \arrow[d, "\pi"'] \arrow[dr, phantom,
"\ulcorner" very near start] & E' \arrow[d, "\pi'"] \\
B \arrow[r, "f"']                                       & B'                  
\end{tikzcd}
\]
which preserve the number of boxes in each row --- we find that such a morphism
is both a co- and contravariant morphism of bundles. It is well known that such
\emph{cartesian} morphisms of bundles correspond to \emph{cartesian} natural
transformations between polynomial functors (cite) --- those whose naturality
squares are pullbacks. This is true as well for Dirichlet functors.
\begin{thm}
A natural transformation $D \to D'$ of Dirichlet functors is Cartesian if and
only if the corresponding bundle map
\[
  \begin{tikzcd}
E \arrow[r, "\term{tot}(f_{\sharp})"] \arrow[d, "\pi"'] \arrow[dr, phantom,
"\ulcorner" very near start] & E' \arrow[d, "\pi'"] \\
B \arrow[r, "f"']                                       & B'                  
\end{tikzcd}
\]
is a pullback. As a corollary, we have an equivalence of categories
$$\Poly_{\ulcorner} \simeq \Dir_{\ulcorner}$$
between polynomial functors with cartesian natural transfromations and Dirichlet
functors with cartesian natural transfromations.
\end{thm}

This story occupies Section \ref{sec:set.level} of our paper.

Polynomial functors play another very important role: they are \emph{generating
  functors} for flat combinatorial species. That is, we have a (non-canonical)
identification
$$P(X) \cong \sum_{n \in \Nb} C_n \times X^n/n!$$
for sets $C_n$ on which the symmetric groups $\Aut(\underline{n})$ act freely
(for \emph{finitary} $P$, commuting with filtered colimits).
The generating functors for general combinatorial species --- the \emph{analytic
functors} --- are not in general polynomial due to a loss of information when
quotienting out by the action of the symmetric groups. This can be fixed by
instead taking \emph{homotopy quotients} of these actions.

In other words, we need to move to homotopy types to make sense of the
fractional coefficients that arise in the exponential power series form
$$\sum_{n \in \Nb} a_n \frac{x^n}{n!}.$$
As we will see, moving to homotopy types also allows us to express the negative
exponent in the Dirichlet series form
$$\sum_{n \in \Nb} a_n n^{-s}.$$

In \emph{$\infty$-Operads as Analytic Monads} \cite{GHK:Analytic.Monads},
Gepner, Haugseng, and Kock take this approach and develop the theory of
polynomial functors on homotopy types. In this setting, being analytic is a
property of a polynomial functors, and is equivalent to admitting a cartesian
map to the functor $\type{Sym}$ given by
$$X \mapsto \sum_{n : \Nb} X^n \sslash n!$$

GHK prove $\infty$-categorical analogues of the above Theorem
\ref{thm:polynomial.set.characterization}.
\begin{thm}[\cite{GHK:Analytic.Monads}]\label{thm:polynomial.type.characterization}
Let $P : \Ha \to \Ha$ be a functor on the $\infty$-category of homotopy types.
Then the following are equivalent:
  \begin{enumerate}
  \item $P$ is polynomial.
  \item There is a bundle $\pi : E \to B$ of homotopy types together with a natural equivalence
    $$P(X) \simeq \sum_{b : B} X^{E_b}.$$
    Or, equivalently, a natural equivalence of $P$ with the composite
    $$\Ha \xto{E^{\ast}} \Ha_{/E} \xto{\pi_!} \Ha_{/B} \xto{B_\ast} \Ha.$$
  \item $P$ is accessible and preserves weakly contractible limits.
  \end{enumerate}
\end{thm}

Note that while this is a strong analogue, it is not a direct generalization:
not all connected limits are weakly contractible.

In Section \ref{sec:type.level}, we will prove analogues of Theorems
\ref{thm:dirichlet.set.characterization} and
\ref{thm:dirichlet.set.equivalence}. In particular, we will prove the following.

\begin{thm}\label{thm:dirichlet.type.characterization}
Let $D : \Set\op \to \Set$ be a contravariant functor. Then the following are
equivalent.
\begin{enumerate}
\item $D$ is Dirichlet.
\item There is a bundle $\pi : E \to B$ of homotopy types together with a natural equivalence
  $$D(X) \cong \sum_{b \in B} E_b^X.$$
  Or, equivalently, a natural equivalence of $D$ with the composite
  $$\Ha\op \xto{(B^{\ast})\op} (\Ha_{/B})\op \xto{[-, \pi]} \Ha_{/B}
  \xto{B_{\ast}} \Ha.$$
\item $D$ preserves weakly contractible limits.
\end{enumerate}
\end{thm}

In Section \ref{sec:ordinary.dirichlet.functors}, we will introduce the notion
of an \emph{ordinary} Dirichlet functor, a categorification of a Dirichlet
series
$$\sum_{n \in \Nb} a_n n^{-s}.$$
In the last installment of his blog \emph{This Week's Finds} (cite), Baez (joint with
Dolan) presents a functor of groupoids categorifying the Riemann Zeta function.
In later, unpublished work (cite), Baez and Dolan give an analogous
categorification of Hasse-Weil Zeta functions. We will recall some facts about
homotopy cardinality and construct these Zeda functors as ordinary Dirichlet
functors in our setting. We leave the investigation of Euler product formulas
and functional equations on the objective level to future work.

\begin{acknowledgements}

\end{acknowledgements}


\section{Dirichlet Functors on Sets} \label{sec:set.level}

Before diving in to the theory of Dirichlet functors, let's first consider the
category $\Set^{\downarrow}$ of bundles of sets.

\begin{prop}
  There is an adjoint quintuple:
  \[
    \begin{tikzcd}
      \Set \arrow[r, "\const"] \arrow[r, leftarrow, shift left = 5,
      "\term{cod}"] \arrow[r, leftarrow, shift right = 5, "\term{dom}"]
      \arrow[r, shift left = 10, "!^{(-)}"]  \arrow[r, shift right = 10, "!_{(-)}"] & \Set^{\downarrow}
    \end{tikzcd}
  \]
  All of the functors $\Set \to \Set^{\downarrow}$ are fully faithful.
\end{prop}
\begin{proof}
There is an adjoint triple $\begin{tikzcd} \ast \arrow[r,leftarrow] \arrow[r,
  bend left] \arrow[r, bend right] & \downarrow \end{tikzcd}$. The middle three
functors are given by restricting along this adjoint triple; the outer two are
given by Kan extending, or more concretely:
  $$X \mapsto \begin{tikzcd} 0 \arrow[d, "!^X"] \\ X \end{tikzcd}$$
  $$X \mapsto \begin{tikzcd} X \arrow[d, "!_X"] \\ \ast \end{tikzcd}$$
It is easy to see that the unit $X \to \term{dom} !_{X}$ is an isomorphism (it is in
fact an identity), so $!_{(-)}$ is fully faithful. Therefore, all the functors
going from $\Set \to \Set^{\downarrow}$ are fully faithful.
\end{proof}

We prove a quick fact we will use later.
\begin{lem}\label{lem:bang.preserves.connected.colimits}
  The functor $!_{(-)}$ preserves connected colimits.
\end{lem}
\begin{proof}
  To see that $!_{(-)}$ preserves connected colimits, recall that colimits in $\Set^{\downarrow}$ are calculated pointwise. It remains to show, then, that the map
  \[
    \begin{tikzcd}
    \colim S_i \arrow[r, equals] \arrow[d] & \colim S_i \arrow[d] \\
    \colim \ast \arrow[r] & \ast 
    \end{tikzcd}
  \]
  is an isomorphism in $\Set^{\downarrow}$, for which it suffices to show that
  $\colim \ast$ is terminal. But the colimit of the terminal diagram is the set
  of connected components of its indexing diagram -- since we assumed the
  indexing diagram was connected, this contains a single element.
\end{proof}

Now we will define the \emph{extent} of a bundle, the Dirichlet functor it
corresponds to. This sends a bundle
$\pi : E \to B$ to the functor
$$X \mapsto \sum_{b \in B} E_b^X.$$
We will, however, give a more abstract definition of the extent, and then
calculate a number of presentations of it.

\begin{defn}
  The functor $!_0 : \downarrow \to \Set\op$ picking out the terminal morphism
  $!_0 : 0 \to 1$ by precomposition an evaluation functor
  $$\Set^{\downarrow} \xleftarrow{\ev_{!_0}} \Fun(\Set\op, \Set).$$
  This functor admits a right adjoint by right Kan extension; we define the
  \emph{Dirichlet extent} functor $\ext : \Set^{\downarrow} \to \Fun(\Set\op,
  \Set)$ to be this right adjoint:
  $$\ext_{\pi} :\equiv \term{ran}_{!_0} \pi.$$
\end{defn}

\begin{prop}
  The functor $\ext$ is fully faithful, giving an equivalence
  $$\Set^{\downarrow} \simeq \Dir$$
  where $\Dir$ is the category of Dirichlet functors and natural transformations.
\end{prop}
\begin{proof}
Since $!_0 : \downarrow \to \Set\op$ is fully faithful, the counit of the
$\ev_{!_0} \vdash \ext$ adjunction --- which is the universal cell defining the
right Kan extension --- is an isomorphism. 
\end{proof}

\begin{prop}\label{prop:set.characterizing.extent}
  Let $\pi : E \to B$ be a bundle. The following are equivalent:
  \begin{enumerate}
  \item The extent $\ext_{\pi} : \Set\op \to \Set$ of $\pi$.
  \item The functor
    $$S \mapsto \sum_{b \in B} E_b^S,$$
    or equivalently the composite
  $$\Set\op \xto{(B^{\ast})\op} (\Set_{/B})\op \xto{[-, \pi]} \Set_{/B}
  \xto{B_{\ast}} \Set.$$
\item The restricted representable functor
  $\Set^{\downarrow}(!_{(-)}, \pi)$.
\item The functor
  $$S \mapsto \lim(\Hom_{\Set}(!_0, S) \to \downarrow \xto{\pi} \Set)$$
  where $\Hom_{\Set}(!_0, S)$ is the slice category of $!_0 : \downarrow \to
  \Set$ over $S$.
\item The functor
  $$S \mapsto \lim(S^{\triangleleft} \xto{!^{\triangleleft}} \ast^{\triangleleft}
  \xto{\pi} \Set)$$
  where $(-)^{\triangleleft}$ is the \emph{left cone} 2-functor, adjoining an initial object.
  \end{enumerate}
\end{prop}
\begin{proof}
We have presented these results in order of most understandable to most
computational; we will prove it a somewhat opposite order.

First, we note that 
$$S \mapsto \lim(\Hom_{\Set\op}(S, !_0) \to \downarrow \xto{\pi} \Set)$$
is the standard right Kan extension formulat for $\ext_{\pi} \equiv
\term{ran}_{!_0} \pi$. Noting that $\Hom_{\Set}(!_0, S) = \Hom_{\Set\op}(S,
!_0)$, we see that (1) is equivalent to (4). Furthermore, noting that
$\Hom_{\Set}(!_0, S)$ simply adjoins an initial object to (the discrete
category) $S$, we see that (4) is equivalent to (5).

Now, we note that every set $S$ is the colimit of the diagram $S^{\triangleleft}
\xto{\ast^{\triangleleft}} \xto{!_0} \Set$, namely:
\[
% https://tikzcd.yichuanshen.de/#N4Igdg9gJgpgziAXAbVABwnAlgFyxMJZABgBoBGAXVJADcBDAGwFcYkRyQBfU9TXfIRTkK1Ok1btOPPtjwEiAZlE0GLNog7deIDHMFEALCvHqp22QIUoATCbWTNAHScBjKBBwIZu-vKHIdjZiDhogxBa++taBpMQhEmEAGtxiMFAA5vBEoABmAE4QALZIAKw0OBBIxD4FxWUVVYjktYUliOUglUg2rfUdjUiKfe1kXU0AbCNIIuNIUzp17XZziAt5bUODa1yUXEA
\begin{tikzcd}
              &              & S                                               &              &               \\
1 \arrow[rru] & 1 \arrow[ru] & \cdots                                          & 1 \arrow[lu] & 1 \arrow[llu] \\
              &              & 0 \arrow[llu] \arrow[lu] \arrow[ru] \arrow[rru] &              &              
\end{tikzcd}
\]
Since, by Lemma \ref{lem:bang.preserves.connected.colimits}, $!_{(-)}$
preserves connected colimits, {\color{red} TODO}.
\end{proof}

Now we are ready to intrinsically characterize the Dirichlet functors.
\begin{thm}
A functor $D : \Set\op \to \Set$ is Dirichlet if and only if it preserves
connected limits. 
\end{thm}
\begin{proof}
First, we show that Dirichlet functors preserve connected limits. If $D$ is
Dirichlet, then by Prop \ref{prop:set.characterizing.extent}, $D$ is naturally
isomorphic to the composite
$$\Set\op \xto{(B^{\ast})\op} (\Set_{/B})\op \xto{[-, \pi]} \Set_{/B}
\xto{B_\ast} \Set.$$
As $(B^{\ast})\op$ is a right adjoint, it commutes with all limits, and as $[-,\pi]$
is part of a multivariable adjunction, it also commutes with all limits.
Finally, we note that $B_{\ast}$ commutes with connected limits.

Now, we show that if $D$ commutes with connected limits, then the unit
$D \to \ext_{D!_0}$ is an isomorphism. By Prop
\ref{prop:set.characterizing.extent}, 
  $$\ext_{D!_0}(S) = \lim(S^{\triangleleft} \xto{!^{\triangleleft}} \ast^{\triangleleft}
  \xto{D!_0} \Set).$$
  
Every set is the connected colimit of the diagram $S^{\triangleleft} \to
\downarrow \xto{!_0} \Set$, and therefore if $D$ preserves this limit, then
$D(S)$ is precisely the above limit $\ext_{D!_0}(S)$.
\end{proof}

Next, we turn to cartesian morphisms.
\begin{prop}
A natural transformation $\phi : D \to D'$ between Dirichlet functors is
cartesian if and only if the induced bundle map $D!_0 \to D'!_0$ is a pullback.
\end{prop}
\begin{proof}
TODO
\end{proof}

\begin{thm}
  There is an equivalence of categories
  $$\type{Poly}_{\ulcorner} \simeq \Dir_{\ulcorner}$$
  between the category of polynomial functors and cartesian transformations and
  Dirichlet functors and cartesian transformations. This equivlalence sends
  representables to representables
  $$(-)^N \mapsto N^{(-)}.$$
\end{thm}
\begin{proof}
TODO
\end{proof}

\section{Dirichlet Functors on Homotopy Types} \label{sec:type.level}

\begin{prop}
  There is an adjoint quintuple:
  \[
    \begin{tikzcd}
      \Ha \arrow[r, "\const"] \arrow[r, leftarrow, shift left = 5,
      "\term{cod}"] \arrow[r, leftarrow, shift right = 5, "\term{dom}"]
      \arrow[r, shift left = 10, "!^{(-)}"]  \arrow[r, shift right = 10, "!_{(-)}"] & \Ha^{\downarrow}
    \end{tikzcd}
  \]
  All of the functors $\Ha \to \Ha^{\downarrow}$ are fully faithful.
\end{prop}
\begin{proof}
There is an adjoint triple $\begin{tikzcd} \ast \arrow[r,leftarrow] \arrow[r,
  bend left] \arrow[r, bend right] & \downarrow \end{tikzcd}$. The middle three
functors are given by restricting along this adjoint triple; the outer two are
given by Kan extending. By Lemma \ref{lem:kan.extending.cone.adds.initial}, we
may calculate these as
  $$X \mapsto \begin{tikzcd} 0 \arrow[d, "!^X"] \\ X \end{tikzcd}$$
  $$X \mapsto \begin{tikzcd} X \arrow[d, "!_X"] \\ \ast \end{tikzcd}$$
It is easy to see that the unit $X \to \term{dom} !_{X}$ is an equivalence (it is in
fact an identity), so $!_{(-)}$ is fully faithful. Therefore, all the functors
going from $\Ha \to \Ha^{\downarrow}$ are fully faithful.
\end{proof}


\begin{prop}\label{lem:bang.preserves.weakly.contractible.colimits}
  The functor $!_{(-)}$ preserves weakly contractible colimits.
\end{prop}
\begin{proof}
  To see that $!_{(-)}$ preserves weakly contractible colimits, recall that colimits in $H^{\downarrow}$ are calculated pointwise. It remains to show, then, that the map
  \[
    \begin{tikzcd}
    \colim S_i \arrow[r, equals] \arrow[d] & \colim S_i \arrow[d] \\
    \colim \ast \arrow[r] & \ast 
    \end{tikzcd}
  \]
  is an equivalence in $H^{\downarrow}$, for which it suffices to show that $\colim \ast$ is contractible. But the colimit of the terminal diagram is the (Kan-fibrant replacement of) its indexing simplicial set -- since we assumed the diagram was weakly contractible, $\colim \ast$ is contractible.
\end{proof}


Now we will define the \emph{extent} of a bundle, the Dirichlet functor it
corresponds to. This sends a bundle
$\pi : E \to B$ to the functor
$$X \mapsto \sum_{b : B} E_b^X.$$
We will, however, give a more abstract definition of the extent, and then
calculate a number of presentations of it.

\begin{defn}
  The functor $!_0 : \downarrow \to \Ha\op$ picking out the terminal morphism
  $!_0 : 0 \to 1$ by precomposition an evaluation functor
  $$\Ha^{\downarrow} \xleftarrow{\ev_{!_0}} \Fun(\Ha\op, \Ha).$$
  This functor admits a right adjoint by right Kan extension; we define the
  \emph{Dirichlet extent} functor $\ext : \Ha^{\downarrow} \to \Fun(\Ha\op,
  \Ha)$ to be this right adjoint:
  $$\ext_{\pi} :\equiv \term{ran}_{!_0} \pi.$$
\end{defn}

\begin{prop}
  The functor $\ext$ is fully faithful, giving an equivalence
  $$\Ha^{\downarrow} \simeq \Dir$$
  where $\Dir$ is the category of Dirichlet functors and natural transformations.
\end{prop}
\begin{proof}
Since $!_0 : \downarrow \to \Ha\op$ is fully faithful, the counit of the
$\ev_{!_0} \vdash \ext$ adjunction --- which is the universal cell defining the
right Kan extension --- is an equivalence by Proposition 9.4.4 of \cite{RV:Elements}. 
\end{proof}

\begin{prop}\label{prop:set.characterizing.extent}
  Let $\pi : E \to B$ be a bundle. The following are equivalent:
  \begin{enumerate}
  \item The extent $\ext_{\pi} : \Ha\op \to \Ha$ of $\pi$.
  \item The functor
    $$S \mapsto \sum_{b : B} E_b^S,$$
    or equivalently the composite
  $$\Ha\op \xto{(B^{\ast})\op} (\Ha_{/B})\op \xto{[-, \pi]} \Ha_{/B}
  \xto{B_{\ast}} \Ha.$$
\item The restricted representable functor
  $\Ha^{\downarrow}(!_{(-)}, \pi)$.
\item The functor
  $$S \mapsto \lim(\Hom_{\Ha}(!_0, S) \to \downarrow \xto{\pi} \Ha)$$
  where $\Hom_{\Ha}(!_0, S)$ is the slice category of $!_0 : \downarrow \to
  \Ha$ over $S$.
\item The functor
  $$S \mapsto \lim(S^{\triangleleft} \xto{!^{\triangleleft}} \ast^{\triangleleft}
  \xto{\pi} \Ha)$$
  where $(-)^{\triangleleft}$ is the \emph{left cone}.
  \end{enumerate}
\end{prop}
\begin{proof}
We have presented these results in order of most understandable to most
computational; we will prove it a somewhat opposite order.

First, we note that 
$$S \mapsto \lim(\Hom_{\Ha\op}(S, !_0) \to \downarrow \xto{\pi} \Ha)$$
is the right Kan extension formula for $\ext_{\pi} \equiv
\term{ran}_{!_0} \pi$ --- see Corollary 9.5.3 of \cite{RV:Elements}. Noting that $\Hom_{\Ha}(!_0, S) = \Hom_{\Ha\op}(S,
!_0)$, we see that (1) is equivalent to (4). Furthermore, noting that
$\Hom_{\Ha}(!_0, S)$ simply adjoins an initial object to (the discrete
category) $S$, we see that (4) is equivalent to (5).

Now, we note that every set $S$ is the colimit of the diagram $S^{\triangleleft}
\xto{\ast^{\triangleleft}} \xto{!_0} \Ha$. Since, by Lemma \ref{lem:bang.preserves.weakly.contractible.colimits}, $!_{(-)}$
preserves weakly contractible colimits, {\color{red} TODO}.
\end{proof}

Now we are ready to intrinsically characterize the Dirichlet functors.
\begin{thm}
A functor $D : \Ha\op \to \Ha$ is Dirichlet if and only if it preserves
weakly contractible limits. 
\end{thm}
\begin{proof}
First, we show that Dirichlet functors preserve connected limits. If $D$ is
Dirichlet, then by Prop \ref{prop:set.characterizing.extent}, $D$ is naturally
equivalent to the composite
$$\Ha\op \xto{(B^{\ast})\op} (\Ha_{/B})\op \xto{[-, \pi]} \Ha_{/B}
\xto{B_\ast} \Ha.$$
As $(B^{\ast})\op$ is a right adjoint, it commutes with all limits, and as $[-,\pi]$
is part of a multivariable adjunction, it also commutes with all limits.
Finally, we note that $B_{\ast}$ commutes with connected limits.

Now, we show that if $D$ commutes with connected limits, then the unit
$D \to \ext_{D!_0}$ is an equivalence. By Prop
\ref{prop:set.characterizing.extent}, 
  $$\ext_{D!_0}(S) = \lim(S^{\triangleleft} \xto{!^{\triangleleft}} \ast^{\triangleleft}
  \xto{D!_0} \Ha).$$
  
Every set is the connected colimit of the diagram $S^{\triangleleft} \to
\downarrow \xto{!_0} \Ha$, and therefore if $D$ preserves this limit, then
$D(S)$ is precisely the above limit $\ext_{D!_0}(S)$.
\end{proof}

Next, we turn to cartesian morphisms.
\begin{prop}
A natural transformation $\phi : D \to D'$ between Dirichlet functors is
cartesian if and only if the induced bundle map $D!_0 \to D'!_0$ is a pullback.
\end{prop}
\begin{proof}
TODO
\end{proof}

\begin{thm}
  There is an equivalence of $\infty$-categories
  $$\type{Poly}_{\ulcorner} \simeq \Dir_{\ulcorner}$$
  between the category of polynomial functors and cartesian transformations and
  Dirichlet functors and cartesian transformations. This equivlalence sends
  representables to representables
  $$(-)^N \mapsto N^{(-)}.$$
\end{thm}
\begin{proof}
TODO
\end{proof}


\section{Ordinary Dirichlet Functors} \label{sec:ordinary.dirichlet.functors}






\section{Appendix}

\begin{lem}\label{lem:cone.initial.equivalence}
Let $\Ca$ be an $\infty$-category with an initial object. For any simplicial set
$J$, let $\Ca^{J^{\triangleleft}}_{\ast}$ be the $\infty$-category of
  $J^{\triangleleft}$-shaped diagrams whose apex is at the initial object; that
  is, the following pullback:
  \[
    \begin{tikzcd}
      \Ca^{J} \arrow[drr, bend left] \arrow[ddr, bend right] \arrow[dr,
      dotted, "\sim"] & & \\
      & \Ca^{J^{\triangleleft}}_{\ast} \arrow[d] \arrow[r] \arrow[dr, phantom,
      "\ulcorner" very near start] & \Ca^{J^{\triangleleft}} \arrow[d,
      "\term{ev}_{\term{apex}}"] \\
      & \ast \arrow[r, "0"'] & \Ca
    \end{tikzcd}
  \]
The the dotted extension map  $\epsilon : \Ca^{J} \to \Ca^{J^{\triangleleft}}_{\ast}$ is an equivalence.
\end{lem}

\begin{lem}\label{lem:kan.extending.cone.adds.initial}
If $\Ca$ is an $\infty$-category with an initial object, then the restriction
$\Ca^{J^{\triangleleft}} \to \Ca^{J}$ has a left adjoint right inverse $\epsilon
: \Ca^J \to \Ca^{J^{\triangleleft}}$ extending a diagram by adjoining the
initial object. 
\end{lem}
\begin{proof}

\end{proof}


\begin{lem}
If $\Ca$ is an $\infty$-category with an initial object $0$, and $d : D \to
\Ca^J$ a family of diagrams with colimit $\colim d : D \to \Ca$, then the family
$D \xto{d} \Ca^J \xto{\epsilon} \Ca^{J^{\triangleleft}}$ has colimit $\colim d$.
\end{lem}
\begin{proof}
For $\colim d : D \to \Ca$ to be the colimit of $d : D \to \Ca^J$ means that 
\end{proof}

