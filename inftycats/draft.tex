
\section{Introduction}
A polynomial functor $P : \Set \to \Set$ is a sum of representables
\begin{equation}\label{eqn.poly}
P(X) := \sum_{b \in B} X^{E_b}
\end{equation}
where the family of sets $E_b$ depends on $b\in B$. This data is known
in the computer science literature as a ``container'', but since such an indexed
family of sets can be represented by as a function
\[
  \begin{tikzcd}
    E \arrow[d, "\pi"] \\
    B
  \end{tikzcd}
\]
we will refer to it as a \emph{bundle}. We think of a bundle as a categorified
Young diagram:

\[
[image]
\]

$B$ is the set of rows and $E_b$ is the set of boxes in row $b$ (possibly empty,
for generalities sake). We can see the polynomial functor \eqref{eqn.poly}
as sending a set $X$ to the set of ways to label a row of the correponding Young
diagram with elements of $X$.

Remarkably, all natural transformations between polynomial functors can be
represented in terms of their associated bundles. A natural transformation $P
\to P'$ corresponds to a pair of functions%
\footnote{We use dependent type syntax, in particular Idris$^?$.}
\[
\left( f : B \to B',\, f^{\sharp} : \dprod{b : B} E'_{fb} \to E_b \right)
\]
noting that $f$ acts covariantly on \dis{columns, right?} and $f^\sharp$ acts contravariantly on
boxes in each row. We refer to these special spans as \dis{can we say \emph{bivariant?}} \emph{contravariant} morphisms of bundles.
In terms of bundles, this is a diagram of the following sort:
\[
  \begin{tikzcd}
    E \arrow[d, "\pi"'] \arrow[r, leftarrow, "f_{\sharp}"] & \bullet \arrow[d] \arrow[r]
    \arrow[dr, phantom, "\ulcorner" very near start] & E' \arrow[d, "\pi'"] \\
    B \arrow[r, equals] & B \arrow[r, "f"'] & B'
  \end{tikzcd}
\]

\begin{thm}[cite]
The category of polynomial functors and natural transformations is equivalent to
the category of bundles and contravariant morphisms.
\end{thm}

This begs the question: what, then, are we to make of the more obvious,
\emph{\dis{univariant?}} morphisms of bundles
\[
  \begin{tikzcd}[column sep=large]
E \arrow[r, "\term{tot}(f_{\sharp})"] \arrow[d, "\pi"'] & E' \arrow[d, "\pi'"] \\
B \arrow[r, "f"']                                       & B'                  
\end{tikzcd}
\]
for which $f$ is covariant in the rows and $f_\sharp$ is again covariant
in each row.
%which, in terms of Young diagrams, are covariant on both the rows and the boxes
%in each row?

It turns out that these covariant maps of bundles correspond to natural
transformations between the appropriate sums of \emph{contravariant}
representables:
$$D(X) := \sum_{b \in B} E_b^X$$
In terms of Young diagrams, such a functor takes a set $X$ to the set of ways to
place the elements of $X$ in the boxes of a row.

Polynomial functors as in \eqref{eqn.poly} get their name from the case in which $B$ and $E$ are finite sets. Consider the function $\term{card}~E_{(-)} : B \to \Nb$, which takes the cardinality of each column $E_b$. Letting $B_n := (\term{card}E_{(-)})\inv(n)$, i.e.\ $B_n$ is the set of columns with $n$ elements,
we find that the for any set $X$, the cardinality
$$|P(X)| = \sum_{n \in \Nb} |B_n||X|^n$$
is a polynomial in the cardinality of $X$. Similarly,
$$|D(X)| = \sum_{n \in \Nb} |B_n|n^{|X|}$$
resembles a Dirichlet series in the cardinality of $X$. Accordingly, we call such
sums of contravariant representables \emph{Dirichlet functors}. Readers who prefer more traditional algebraic notation can see \cite{SpivakMyers}, which was the set-theoretic inspiration for the present paper.\dis{Agree? Maybe there's a better place to put this?}

We will show in this paper that Dirichlet functors are, quite robustly, the
contravariant analogue of polynomial functors. In particular, the many equivalent ways to
say that a functor is polynomial have contravariant analogues.
\begin{thm}[cite]\label{thm:polynomial.set.characterization}
  Let $P : \Set \to \Set$ be a functor. Then the following are equivalent.
  \begin{enumerate}
  \item $P$ is polynomial.
  \item $P$ is the sum of covariant representables.
  \item There is a bundle $\pi : E \to B$ together with a natural isomorphism
    $$P(X) \cong \sum_{b \in B} X^{E_b}.$$
    Or, equivalently, a natural isomorphism of $P$ with the composite
    $$\Set \xto{\dis{\Delta_{!_E}}} \Set_{/E} \xto{\dis{\Pi_\pi}} \Set_{/B} \xto{\dis{\Sigma_{!_B}}} \Set.$$
  \item $P$ is accessible and preserves connected limits.
  \end{enumerate}
\end{thm}

Analogously, we will prove the following theorem.
\begin{thm}\label{thm:dirichlet.set.characterization}
Let $D : \Set\op \to \Set$ be a contravariant functor. Then the following are
equivalent.
\begin{enumerate}
\item $D$ is Dirichlet.
\item $D$ is the sum of contravariant representables.
\item There is a bundle $\pi : E \to B$ together with a natural isomorphism
  $$D(X) \cong \sum_{b \in B} E_b^X.$$
  Or, equivalently, a natural isomorphism of $D$ with the composite
  $$\Set\op \xto{(\Delta_{!_B})\op} (\Set_{/B})\op \xto{[-, \pi]} \Set_{/B}
  \xto{\Sigma_{!_B}} \Set.$$
\item $D$ preserves connected limits.
\end{enumerate}
\end{thm}

Note that we no longer need to assume accessiblity. This is a general feature of
the theory of Dirichlet functors; it is a bit ``smaller'' and more manageable
than that of polynomials. In particular, a Dirichlet functor is determined by
its action on the terminal morphism $!_0 : 0 \to \ast$ of the empty set \dis{also the terminal twisted arrow; any relation to GHK?}. As a corollary, Dirichlet functors form a topos.

\begin{thm}\label{thm:dirichlet.set.equivalence}
The functor $\Set^{\down} \to \type{Fun}(\Set\op, \Set)$, given by sending $\pi : E
\to B$ to the induced Dirichlet functor $X \mapsto \sum_{b \in B} E_b^X$, is
fully faithful, and so gives an equivalence
$$\Set^{\down} \simeq \Dir$$
between the topos of bundles and the category of Dirichlet functors, with inverse given by evalutation at
$!_0 : 0 \to \ast$.
\end{thm}


Now, object-wise, a Dirichlet functor and a polynomial functor are determined by
the same data --- a bundle $\pi : E \to B$ of sets. Accordingly, one would
expect for any set $N$ a transformation
$$X^N \mapsto N^X$$
turning polynomial functors into Dirichlet functors, and vice versa. But the
natural transformations between each sort of functor induce different morphisms
between the bundles; natural transformations between polynomial functors induce
contravariant bundle morphisms, while natural transformations between Dirichlet
functors induce covariant bundle morphisms. However, if we restrict to those
morphisms of bundles which are \emph{isovariant} on the fibers --- that is, the
pullback diagrams of the form
\[
  \begin{tikzcd}[column sep=large]
E \arrow[r, "\term{tot}(f_{\sharp})"] \arrow[d, "\pi"'] \arrow[dr, phantom,
"\ulcorner" very near start] & E' \arrow[d, "\pi'"] \\
B \arrow[r, "f"']                                       & B'                  
\end{tikzcd}
\]
which preserve the number of boxes in each row --- we find that such a morphism
is both a co- and contravariant morphism of bundles. It is well known that such
\emph{cartesian} morphisms of bundles correspond to \emph{cartesian} natural
transformations between polynomial functors (cite) --- those whose naturality
squares are pullbacks. This is true as well for Dirichlet functors.
\begin{thm}
A natural transformation $D \to D'$ of Dirichlet functors is Cartesian if and
only if the corresponding bundle map
\[
  \begin{tikzcd}[column sep=large]
E \arrow[r, "\term{tot}(f_{\sharp})"] \arrow[d, "\pi"'] \arrow[dr, phantom,
"\ulcorner" very near start] & E' \arrow[d, "\pi'"] \\
B \arrow[r, "f"']                                       & B'                  
\end{tikzcd}
\]
is a pullback. As a corollary, we have an equivalence of categories
$$\type{Poly}_{\ulcorner} \simeq \Dir_{\ulcorner}$$
between polynomial functors with cartesian natural transfromations and Dirichlet
functors with cartesian natural transfromations.
\end{thm}

This story occupies Section \ref{sec:set.level} of our paper.

Polynomial functors play another very important role: they are \emph{generating
  functors} for flat combinatorial species. That is, we have a (non-canonical)
identification
$$P(X) \cong \sum_{n \in \Nb} C_n \times X^n/n!$$
for sets $C_n$ on which the symmetric groups $\Aut(\underline{n})$ act freely
(for \emph{finitary} functors $P$, i.e.\ those that commute with filtered colimits).
The generating functors for general combinatorial species --- the \emph{analytic
functors} --- are not in general polynomial due to a loss of information when
quotienting out by the action of the symmetric groups. This can be fixed by
instead taking \emph{homotopy quotients} of these actions.

In other words, we need to move to homotopy types to make sense of the
fractional coefficients that arise in the exponential power series form
$$\sum_{n \in \Nb} a_n \frac{x^n}{n!}.$$
As we will see, moving to homotopy types also allows us to express the negative
exponent in the Dirichlet series form
$$\sum_{n \in \Nb} a_n n^{-s}.$$

In \emph{$\infty$-Operads as Analytic Monads} \cite{GHK:Analytic.Monads},
Gepner, Haugseng, and Kock take this approach and develop the theory of
polynomial functors on homotopy types. In this setting, being analytic is a
property of \dis{some?} polynomial functors, and is equivalent to admitting a cartesian
map to the functor $\type{Sym}$ given by
$$\type{Sym}X := \sum_{n : \Nb} X^n \sslash n!$$
\dis{This is the homotopy quotient?}

Let $\Ha$ be the $\infty$-category of homotopy types. GHK prove $\infty$-categorical analogues of the above Theorem
\ref{thm:polynomial.set.characterization}.
\begin{thm}[\cite{GHK:Analytic.Monads}]\label{thm:polynomial.type.characterization}
For any functor $P : \Ha \to \Ha$, the following are equivalent:
  \begin{enumerate}
  \item $P$ is polynomial.
  \item There is a bundle $\pi : E \to B$ of homotopy types together with a natural equivalence
    $$P(X) \simeq \sum_{b : B} X^{E_b}.$$
    Or, equivalently, a natural equivalence of $P$ with the composite
    $$\Ha \xto{E^{\ast}} \Ha_{/E} \xto{\pi_!} \Ha_{/B} \xto{B_\ast} \Ha.$$
  \item $P$ is accessible and preserves weakly contractible limits.
  \end{enumerate}
\end{thm}

Note that while this is a strong analogue, it is not a direct generalization of \cref{thm:polynomial.set.characterization}:
though it is true that in 1-categories all connected limits are weakly contractible, this is not true in general.

In Section \ref{sec:type.level}, we will prove analogues of Theorems
\ref{thm:dirichlet.set.characterization} and
\ref{thm:dirichlet.set.equivalence}. \dis{What analogue of \ref{thm:dirichlet.set.equivalence} are we proving?} In particular, we will prove the following.

\begin{thm}\label{thm:dirichlet.type.characterization}
Let $D : \Set\op \to \Set$ \dis{Do you mean $\Ha$ here?} be a contravariant functor. Then the following are
equivalent.
\begin{enumerate}
\item $D$ is Dirichlet. \dis{In what sense?}
\item There is a bundle $\pi : E \to B$ of homotopy types together with a natural equivalence
  $$D(X) \cong \sum_{b \in B} E_b^X.$$
  Or, equivalently, a natural equivalence of $D$ with the composite
  $$\Ha\op \xto{(B^{\ast})\op} (\Ha_{/B})\op \xto{[-, \pi]} \Ha_{/B}
  \xto{B_{\ast}} \Ha.$$
\item $D$ preserves weakly contractible limits.
\end{enumerate}
\end{thm}

In Section \ref{sec:ordinary.dirichlet.functors}, we will introduce the notion
of an \emph{ordinary} Dirichlet functor, a categorification of a Dirichlet
series
$$\sum_{n \in \Nb} a_n n^{-s}.$$
In the last installment of his blog \emph{This Week's Finds} (cite), Baez (joint with
Dolan) presents a functor of groupoids categorifying the Riemann Zeta function.
In later, unpublished work (cite), Baez and Dolan give an analogous
categorification of Hasse-Weil Zeta functions. We will recall some facts about
homotopy cardinality and construct these Zeta functors as ordinary Dirichlet
functors in our setting. We leave the investigation of Euler product formulas
and functional equations on the objective level to future work.

\begin{acknowledgements}

\end{acknowledgements}

\section{Dirichlet Functors on Homotopy Types} \label{sec:type.level}

\begin{prop}
  There is an adjoint quintuple:
  \[
    \begin{tikzcd}
      \Ha \arrow[r, "\const"] \arrow[r, leftarrow, shift left = 5,
      "\term{cod}"] \arrow[r, leftarrow, shift right = 5, "\term{dom}"]
      \arrow[r, shift left = 10, "!^{(-)}"]  \arrow[r, shift right = 10, "!_{(-)}"] & \Ha^{\down}
    \end{tikzcd}
  \]
  All three functors $\Ha \to \Ha^{\down}$ are fully faithful.
\end{prop}
\begin{proof}
There is an adjoint triple $\begin{tikzcd} \ast \arrow[r,leftarrow] \arrow[r,
  bend left] \arrow[r, bend right] & \down \end{tikzcd}$. The middle three
functors are given by restricting along this adjoint triple; the outer two are
given by Kan extending. By Lemma \ref{lem:kan.extending.cone.adds.initial}, we
may calculate these as
  $$X \mapsto \begin{tikzcd} 0 \arrow[d, "!^X"] \\ X \end{tikzcd}
  \qquad\text{and}\qquad
  X \mapsto \begin{tikzcd} X \arrow[d, "!_X"] \\ \ast \end{tikzcd}$$
It is easy to see that the unit $X \to \term{dom} !_{X}$ is an equivalence (it is in
fact an identity), so $!_{(-)}$ is fully faithful. Therefore, all the functors
going from $\Ha \to \Ha^{\down}$ are fully faithful.
\end{proof}


\begin{prop}\label{lem:bang.preserves.weakly.contractible.colimits}
  The functor $!_{(-)}$ preserves weakly contractible colimits.
\end{prop}
\begin{proof}
  To see that $!_{(-)}$ preserves weakly contractible colimits, recall that colimits in $H^{\down}$ are calculated pointwise. It remains to show, then, that the map
  \[
    \begin{tikzcd}
    \colim S_i \arrow[r, equals] \arrow[d] & \colim S_i \arrow[d] \\
    \colim \ast \arrow[r] & \ast 
    \end{tikzcd}
  \]
  is an equivalence in $H^{\down}$, for which it suffices to show that $\colim \ast$ is contractible. But the colimit of the terminal diagram is the (Kan-fibrant replacement of) its indexing simplicial set -- since we assumed the diagram was weakly contractible, $\colim \ast$ is contractible.
\end{proof}

\begin{lem}\label{lem:type.bundle.yoneda}
The Yoneda embedding $\yo : \downarrow\op \to \Ha^{\downarrow}$ is equivalent to the
composite $$\downarrow\op \xto{\sigma\op} \downarrow \xto{!_{0}} \Ha
\xto{!_{(-)}} \Ha^{\downarrow}.$$

As a corollary, for any $\pi : \downarrow \to \Ha$, $\down\op \xto{\sigma\op}
\down \xto{\pi} \Ha$ equals the composite
$$\down\op \xto{!_0\op} \Ha\op \xto{!_{(-)}\op}
(\Ha^{\downarrow})\op \xto{\Ha^{\downarrow}(-, \pi)} \Ha$$
by the Yoneda lemma.
\end{lem}
\begin{proof}
  This follows from Lemma \ref{lem:set.bundle.yoneda}, since $\downarrow$ is a $1$-category.
\end{proof}

Now we will define the \emph{extent} of a bundle, the Dirichlet functor it
corresponds to. This sends a bundle
$\pi : E \to B$ to the functor
$$X \mapsto \sum_{b : B} E_b^X.$$
We will, however, give a more abstract definition of the extent, and then
calculate a number of presentations of it.

\begin{defn}
Consider the functor $!_0 : \down \to \Ha\op$ picking out the unique morphism $1\to 0$ in $\Ha\op$. Sending a functor $\Ha\op\to\Ha$ to its precomposition with $!_0$ gives an evaluation functor
  $$\Ha^{\down} \xleftarrow{\ev_{!_0}} \Fun(\Ha\op, \Ha).$$
  This functor admits a right adjoint by right Kan extension; we define the
  \emph{Dirichlet extent} functor $\ext : \Ha^{\down} \to \Fun(\Ha\op,
  \Ha)$ to be this right adjoint:
  $$\ext_{\pi} :\equiv \term{ran}_{!_0\op \circ \sigma} \pi.$$
\end{defn}

\begin{prop}
  The functor $\ext$ is fully faithful, giving an equivalence 
  $$\Ha^{\down} \simeq \Dir$$
  where $\Dir$ is the category of Dirichlet functors and natural transformations.\dis{What is $\Dir$; have we defined Dirichlet functors in this context?}
\end{prop}
\begin{proof}
Since $!_0 : \down \to \Ha\op$ is fully faithful, the counit of the
$\ev_{!_0} \vdash \ext$ adjunction --- which is the universal cell defining the
right Kan extension --- is an equivalence by Proposition 9.4.4 of \cite{RV:Elements}. 
\end{proof}

\begin{prop}\label{prop:set.characterizing.extent}
  Let $\pi : E \to B$ be a bundle. The following are equivalent:
  \begin{enumerate}
  \item The extent $\ext_{\pi} : \Ha\op \to \Ha$ of $\pi$.
  \item The functor
    $$S \mapsto \sum_{b : B} E_b^S,$$
    or equivalently the composite
  $$\Ha\op \xto{(B^{\ast})\op} (\Ha_{/B})\op \xto{[-, \pi]} \Ha_{/B}
  \xto{B_{\ast}} \Ha.$$
\item The following pullback in $\Fun(\Ha\op, \Ha)$: 
\[
\begin{tikzcd}
	\ext_\pi\ar[r]\ar[d]\arrow[dr, phantom,
      "\ulcorner" very near start]&
	B \ar[d]\\
	\Ha(-,E)\ar[r]&
	\Ha(-,B)
\end{tikzcd}
\]
\item The restricted representable functor
  $\Ha^{\down}(!_{(-)}, \pi)$.
\item The functor
  $$S \mapsto \lim(\Hom_{\Ha}(!_0, S)\op \to \down\op \xto{\sigma\op} \down \xto{\pi} \Ha)$$
  where $\Hom_{\Ha}(!_0, S)$ is the slice category of $!_0 : \down \to
  \Ha$ over $S$.
\item The functor
  $$S \mapsto \lim(( S^{\triangleleft} )\op \xto{( !^{\triangleleft} )\op}
  (\ast^{\triangleleft})\op \xto{\sigma\op} \down
  \xto{\pi} \Ha)$$
  where $(-)^{\triangleleft}$ is the \emph{left cone}.
  \end{enumerate}
\end{prop}
\begin{proof}
We have presented these results in order of most understandable to most
computational; we will prove it a somewhat opposite order.


First, we note that 
$$S \mapsto \lim(\Hom_{\Ha\op}(S, !_0\op \circ \sigma) \to \down \xto{\pi} \Ha)$$
is the right Kan extension formula for $\ext_{\pi} \equiv
\term{ran}_{!_0} \pi$ --- see Corollary 9.5.3 of \cite{RV:Elements}. Now,
$\Hom_{\Ha\op}(S, !_0\op \circ \sigma) \simeq \Hom_{\Ha}(!_0 \circ \sigma\op,
S)\op$ over $\down$, which we may see by noting that $(-)\op : {\infty}\Cat \to {\infty}\Cat^{\term{co}}$ is a cosmological equivalence and therefore preserves
comma squares. Furthermore, we have the following equivalence:
\[
% https://tikzcd.yichuanshen.de/#N4Igdg9gJgpgziAXAbVABwnAlgFyxMJZABgBpiBdUkANwEMAbAVxiRAB12AJCAWwH1gnLnQC+ACgCE-YgAJOAYywAnBfPbYA5rzqcIaUrIDKASj1oQo0uky58hFGQCMVWoxZthfQcLFSZhqbmltYgGNh4BERO5K70zKyIHOxQEADuhFY2EfbRpC7U8R5JnKkZwaKuMFCa8ESgAGbKfEhkIDgQSDEgDHQARjAMAAq2kQ4gyliaABY4IIXuicnYvCGNzauIbR1IAExZIE0tiLvUO4gAzAsJnhpTOmuHG11nnZeVokA
\begin{tikzcd}
{\Hom_{\Ha}(!_0 \circ \sigma\op, S)\op} \arrow[d, "\sim"'] \arrow[r] & \down \arrow[d, "\sigma"] \\
{\Hom_{\Ha}(!_0, S)\op} \arrow[r]                                    & \down\op                 
\end{tikzcd}
\]
and therefore we may equivalently calculate this limit as 
$$\lim(\Hom_{\Ha}(!_0, S)\op \to \down\op \xto{\sigma\op} \down \xto{\pi} \Ha).$$
This gives us the equivalence between (1) and (5). By Lemma
\ref{lem:slice.cone.equiv}, we see also that (5) and (6) are equivalent.

Now, using Lemmas \ref{lem:bang.preserves.weakly.contractible.colimits} and \ref{lem:type.bundle.yoneda}, we may make
the following calculation identifying (4) with (6):

\begin{align*}
  \Ha^{\down}(!_S, \pi) &= \Ha^{\down}(!_{\colim(S^{\triangleleft} \xto{!^{\triangleleft}} \ast^{\triangleleft} \xto{!_0} \Ha)}, \pi) \\
  &\simeq \Ha^{\down}(\colim(S^{\triangleleft} \xto{!^{\triangleleft}} \ast^{\triangleleft} \xto{!_0} \Ha \xto{!_{(-)}} \Ha^{\down}), \pi)\\
&\simeq \lim((S^{\down})\op \xto{(!^{\triangleleft})\op} (\ast^{\triangleleft})\op \xto{!_0\op} \Ha\op \xto{!_{(-)}} (\Ha^{\down})\op \xto{\Ha^{\down}(-, \pi)} \Ha) \\
&\simeq \lim((S^{\down})\op \xto{(!^{\triangleleft})\op} (\ast^{\triangleleft})\op \xto{\sigma\op} \down \xto{\pi} \Ha).
\end{align*}

We see that (3) is equivalent to (4) by Lemma
\ref{lem:hom.type.arrow}.\footnote{\jaz{This applies by using the module
    formalism to deal with this representable functors. I don't know how much I
    should go into this.}}
\end{proof}

Now we are ready to intrinsically characterize the Dirichlet functors.
\begin{thm}
A functor $D : \Ha\op \to \Ha$ is Dirichlet if and only if it preserves
weakly contractible limits. 
\end{thm}
\begin{proof}
First, we show that Dirichlet functors preserve connected limits. If $D$ is
Dirichlet, then by Prop \ref{prop:set.characterizing.extent}, $D$ is naturally
equivalent to the restricted representable functor $\Ha^{\down}(!_{-}, \pi)$.
Since, by Lemma \ref{lem:bang.preserves.weakly.contractible.colimits}, $!_{(-)}$
preserves weakly contractible colimits, $D$ sends weakly contractible colimits
to limits.

Now, we show that if $D$ commutes with weakly contractible limits, then the unit
$D \to \ext_{D!_0}$ is an equivalence. By Prop
\ref{prop:set.characterizing.extent}, 
  $$\ext_{D!_0}(S) = \lim(S^{\triangleleft} \xto{!^{\triangleleft}} \ast^{\triangleleft}
  \xto{D!_0} \Ha).$$
  
Every homotopy type $S$ is the weakly contractible colimit of the diagram $S^{\triangleleft} \to
\down \xto{!_0} \Ha$ by Lemma \ref{lem:every.type.weakly.contractible.colimit}. Therefore if $D$ preserves this limit, then
$D(S)$ is precisely the above limit $\ext_{D!_0}(S)$.
\end{proof}

Next, we turn to cartesian morphisms.
\begin{prop}\label{prop:type.dirichlet.cartesian.iff.bundle.cartesian}
A natural transformation $\phi : D \to D'$ between Dirichlet functors is
cartesian if and only if the induced bundle map $D!_0 \to D'!_0$ is a pullback.

As a corllary, the equivalence $\Dir \simeq \Ha^{\down}$ restricts to an
equivalence
$$\Dir_{\ulcorner} \simeq \Ha^{\down}_{\ulcorner}$$
between Dirichlet functors and cartesian natural transformations and bundles and
pullback sqaures.
\end{prop}
\begin{proof}
We want to show that for any $f\colon D\to D'$ the square
\begin{equation}\label{eqn.type.cart1}
\begin{tikzcd}
	D(1)\ar[r, "f_1"]\ar[d, "\pi"']&
	D'(1)\ar[d, "\pi'"]\\
	D(0)\ar[r, "f_0"']&
	D'(0)
\end{tikzcd}
\end{equation}
is a pullback in $\Set$ iff for all functions $g\colon X\to X'$, the naturality square
\begin{equation}\label{eqn.type.cart2}
\begin{tikzcd}
  D(X')\ar[r, "f_{X'}"]\ar[d, "D(g)"']&
  D'(X')\ar[d, "D'(g)"]\\
  D(X)\ar[r, "f_X"']&
  D'(X)
\end{tikzcd}
\end{equation}
is a pullback in $\Ha$. We will freely use the natural isomorphism
$D(X)\cong\Ha^{\down}(!_X, D!_0)$ from
\ref{prop:set.characterizing.extent}, which allows us to identify Diagram
\ref{eqn.type.cart2} with
\begin{equation}\label{eqn.type.cart3}
\begin{tikzcd}
 \Ha^{\down}(!_{X'}, D!_0)     \ar[r, "f_{!_0,\ast}"]\ar[d, "!_g^{\ast}"']& \Ha^{\down}(!_{X'}, D'!_0)
  \ar[d, "!_{g}^{\ast}"]\\
  \Ha^{\down}(!_{X}, D!_0)\ar[r, "f_{!_0, \ast}"']&
 \Ha^{\down}(!_X, D'!_0) 
\end{tikzcd}
\end{equation}

The square in Diagram \ref{eqn.type.cart1} is a
special case of that in Diagram \ref{eqn.type.cart2}, namely for $g\coloneqq !_0$;
this establishes the only-if direction.

For the if direction, note that Diagram \ref{eqn.type.cart3} is a pullback
whenever Diagram \ref{eqn.type.cart1} is, as vertical maps are orthogonal to
cartesian maps in $\Ha^{\down}$.
\end{proof}

\begin{thm}\label{thm:type.poly.cart.equiv.dir.cart}
  There is an equivalence of $\infty$-categories
  $$\type{Poly}_{\ulcorner} \simeq \Dir_{\ulcorner}$$
  between the category of polynomial functors and cartesian transformations and
  Dirichlet functors and cartesian transformations. This equivlalence sends
  representables to representables
  $$(-)^N \mapsto N^{(-)}.$$
\end{thm}
\begin{proof}
This follows by composing the equivalence of $\Dir_{\ulcorner}$ with
$\Ha^{\down}$ from Prop
\ref{prop:type.dirichlet.cartesian.iff.bundle.cartesian} with that of
Proposition 2.4.13 of \cite{GHK:Analytic.Monads}, noting that
$\Ha^{\down}_{\ulcorner}$ is the category of $(\ast,\ast)$-polynomials. 
\end{proof}

\begin{cor}
  Let $D$ be a Dirichlet functor. Then the $\infty$-category
  $\Dir_{/_{\ulcorner} D}$
  of Dirichlet functors with a cartesian map to $D$ is an $\infty$-topos.
\end{cor}
\begin{proof}
By Theorem \ref{thm:type.poly.cart.equiv.dir.cart}, this $\infty$-category is equivalent
to $\type{Poly}_{/_{\ulcorner} P}$ for a polynomial $P$. But this is an
$\infty$-topos by Theorem 2.6.1 of \cite{GHK:Analytic.Monads}.
\end{proof}

Now, since $\Dir$ is a topos, so is $\Dir_{/D}$. And, as we saw above,
$\Dir_{/_{\ulcorner} D}$ is a topos as well. What is the relationship between
$\Dir_{/D}$ and $\Dir_{/_{\ulcorner} D}$?

We will show that $\Dir_{/_{\ulcorner} D}$ is a subtopos of $\Dir_{/D}$ induced
by a \emph{lex modality} on $\Dir_{/D}$. We will recall the definition of a lex
modality and its basic theory here, before applying it to prove our desired
theorem. Modalities were defined in Homotopy Type Theory in the HoTT Book
\cite{UFP:HoTT.Book}, and their theory developed substantially in
\cite{RSS:Modalities.in.HoTT}. This theory was externalized to
$\infty$-toposes in \cite{ABFJ:Blakers.Massey} and \cite{ABFJ:Goodwillie} as an
efficient way for working with localizations of $\infty$-toposes.

\begin{defn}
A \emph{modality} on an $\infty$-topos $\Ca$ is a stable orthogonal factorization
system $(\La, \Ra)$. This is an orthogonal factorization system in which the
left class is stable under pullback.

A modality is \emph{lex} if for any square
\[
\begin{tikzcd}
A \arrow[r, "h"] \arrow[d, "g"'] & C \arrow[d, "f"] \\
B \arrow[r, "k"']                & D               
\end{tikzcd}
\]
where $f$ and $g$ are in $\La$ and $h$ and $k$ are in $\Ra$ is a pullback. (See
Theorem 3.11 of \cite{RSS:Modalties.in.HoTT}, condition (vii).)


We say that a modality $(\La, \Ra)$ is \emph{contained in} $(\La', \Ra')$ ---
written $(\La, \Ra) \leq (\La', \Ra')$ --- if $\Ra \subseteq \Ra'$.
\end{defn}

\begin{lem}\label{lem:slice.modality}
If $(\La, \Ra)$ is a modality on $\Ca$, then there is an induced modality $(\La_X,
\Ra_X)$ on $\Ca_{/X}$ where $\La_X$ (resp. $\Ra$) consists of those maps in $\Ca_{/X}$
whose image under the forgetful functor to $\Ca$ is in $\La$ (resp. $\Ra$).

If $(\La, \Ra)$ is lex, then so is $(\La_X, \Ra_X)$.
\end{lem}
\begin{proof}
\jaz{That this forms an orthogonal factorization system is claimed without proof
as Lemma 3.1.7 in \cite{ABFJ:Blakers.Massey}. TODO would be to prove this
fully.}

Recall that the forgetful functor $\Ca_{/X} \to \Ca$ creates pullbacks. Therefore,
$(\La_X, \Ra_X)$ is a modality --- $\La_X$ is closed under pullback since $\La$
is. Furthermore, if $(\La, \Ra)$ is lex, then so is $(\La_X, \Ra_X)$ because the
necessary diagram is a pullback if and only if its underlying diagram in $\Ca$
is one. 
\end{proof}

\begin{prop}\label{prop:lex.modalities.subtoposes}
  For any lex modality $(\La, \Ra)$ on an $\infty$-topos $\Ca$, we get a subtopos
  consisting of those objects $X \in \Ca$ for which the terminal map $!_X : X
  \to \ast$ is in $\Ra$. This includes the order of lex modalities into the
  order of subcategories of $\Ca$ whose image consists of precisely the subtoposes.
\end{prop}
\begin{proof}
\jaz{TODO. I
  think I can find a citation for this --- it is certainly folklore. I'll check \cite{RSS:Modalities.in.HoTT}.}
\end{proof}

\begin{lem}\label{lem:cartesian.modality}
The orthogonal factorization system $(\term{vertical}, \term{cartesian})$ on
$\Ha^{\downarrow}$ (given by the cartesian fibration $\Ha\op \to \Ha$) is a
lex modality.
\end{lem}
\begin{proof}
\jaz{TODO: Cite a proof that this is an orthogonal factorization system.}

  Vertical morphisms of bundles are closed under pullback since pullbacks are
computed pointwise and the pullback of an equivalence is an equivalence. The
modality is lex because in the following cube:
\[
    \begin{tikzcd}[row sep=1.5em, column sep = 1.5em]
    A_1 \arrow[rr] \arrow[dr, crossing over] \arrow[dd,swap] \arrow[ddrr, phantom, "\ulcorner"
    very near start] &&
    C_1 \arrow[dd] \arrow[dr] \\
    & B_1 \arrow[rr, crossing over]\arrow[dd, crossing over] \arrow[ddrr,
    phantom, "\ulcorner" very near start]&&
    D_1 \arrow[dd,"h"] \\
    A_2 \arrow[rr,] \arrow[dr, "\sim"'] && C_2 \arrow[dr, "\sim"] \\
    & B_2 \arrow[rr] && D_2 
    \end{tikzcd}
\]

The bottom square is a pullback since its opposite faces are equivalences, and
therefore the top face is also a pullback, so that the cube is a pullback in $\Ha^{\down}$.
\end{proof}

\begin{thm}
The left exact modality $(\term{vertical}_{\pi}, \term{cartesian}_{\pi})$ on
$\Ha^{\downarrow}_{/\pi}$ exhibits $\Ha^{\downarrow}_{/_{\ulcorner} \pi}$ as a
subtopos.

As a corollary, we have a subtopos inclusion $\Dir_{/_{\ulcorner} D}
\hookrightarrow \Dir_{/D}$. 
\end{thm}
\begin{proof}
This follows immediately from Proposition \ref{prop:lex.modalities.subtoposes} by
noting that the objects of $\Ha^{\down}_{/_{\ulcorner} \pi}$ are precisely those objects
in $\Ha^{\down}_{/\pi}$ whose terminal morphism is in $\term{cartesian}_{\pi}$.
\end{proof}

\begin{rmk}
Note that $\const : \Ha \to \Ha^{\downarrow}$ induces an equivalence $\Ha \simeq
\Ha^{\down}_{/_{\ulcorner} \ast}$. Therefore, we see that the above subtopos
inclusion coincides with that of $\const : \Ha \hookrightarrow \Ha^{\down}$ in
the case that $\pi = \ast$. In particular, note that $\Ha^{\down}_{\ulcorner}
\neq \Ha^{\down}_{/_{\ulcorner} \ast}$.
\end{rmk}

\section{Ordinary Dirichlet Functors} \label{sec:ordinary.dirichlet.functors}

In this section, we will categorify Dirichlet series
$$\sum_{n \in \Nb} a_n n^{-s}.$$

First, we review the notion of \emph{homotopy cardinality}.

\begin{defn}
Let $B$ be a homotopy type. We define the \emph{homotopy cardinality} $\#B$ to be
the following sum:
$$\#B := \sum_{[b] \in \pi_0(B)} \prod_{n = 1}^{\infty} |\pi_n(B, b)|^{(-1)^n}$$

We note that
$$\#B = \sum_{[b] \in \pi_0(B)} \frac{1}{\#\Omega(B, b)}$$
\end{defn}

We will leave questions concerning the convergence and well-definedness of the
homotopy cardinality of a type, noting only that if a type $B$ is totally finite in
that it has finitely many connected components, its homotopy groups are finite,
and they eventually vanish, then $\#B$ is a finite sum of a finite product and so
is well defined.

The main lemma concerning homotopy cardinality is its behavior with respect to
fibrations.
\begin{lem}
Let $\pi : E \to B$ be a map of homotopy types, and let $E_b$ denote the fiber
over $b : B$. Then
$$\#E = \sum_{[b] \in \pi_0(B)}\frac{\#E_b}{\#\Omega(B, b)}.$$
As a corollary:
\begin{enumerate}
\item $\#(A \times B) = \#A \cdot \#B$ and so $\#(B^n) = (\#B)^n$.
\item If $B$ is pointed and connected and $F$ is the fiber of $\pi$ over the
  basepoint, then $\#E = \#F \cdot \#B$.
\item If $A_n$ is a family of types for $n : \Nb$, then $\#(\sum_n A_n) = \sum_n \#A_n$.
\end{enumerate}
\end{lem}
\begin{proof}
  We note that by the long exact sequence of homotopy groups, we have that
  $$|\pi_n(E, e)| = |\pi_n(E_{\pi e}, e)|\cdot |\pi_n(B, \pi e)|$$
  Therefore, we may calculate
  \begin{align*}
    \#E &= \sum_{[e] \in \pi_0(E)}\prod_{n=1}^{\infty} |\pi_n(E, e)|^{(-1)^n}\\
    &= \sum_{[b] \in \pi_0(B)} \sum_{[e] \in \pi_0(E_b)} \prod_{n=1}^{\infty} (|\pi_n(E_{b}, e)|\cdot |\pi_n(B, b)|)^{(-1)^n}\\
    &= \sum_{[b] \in \pi_0(B)}\prod_{n=1}^{\infty} |\pi_n(B, b)|^{(-1)^n} \sum_{[e] \in \pi_0(E_b)} \prod_{n=1}^{\infty} |\pi_n(E_{b}, e)|^{(-1)^n}\\
    &= \sum_{[b] \in \pi_0(B)}\frac{\#E_b}{\#\Omega(B, b)}.
  \end{align*}
  
  We may then take various maps $\pi : E \to B$ to recover the corollaries:
  \begin{enumerate}
  \item Take the projection $A \times B \to B$.
  \item Take the map $\pi : E \to B$, noting that if $B$ is pointed connected then $\#B =
    \frac{1}{\Omega B}$.
  \item Take the projection $\sum_n A_n \to \Nb$.
  \end{enumerate}
\end{proof}

Now, we note that if $B$ is connected, then $\#B = \frac{1}{\#\Omega(B, b)}$ for
any point $b : B$. Therefore, if $B$ is connected with finite homs of
cardinality $n$, then $\#B = \frac{1}{n}$ and so $\#(B^s) = n^{-s}$. This is the
observation --- originally due to Baez and Dolan in \emph{This Weeks Finds}
(cite) --- that we will use to categorify Dirichlet series.

\begin{defn}
A homotopy type $B$ is \emph{locally finite and connected}, or \emph{lfc}, if it
is $0$-connected and its homs are finite.

We denote by $\type{lfc}$ the homotopy type of lfc types, and note that the
bundle $\fst : \type{lfc}_{\ast} \to \type{lfc}$ forgetting the point of a
pointed lfc type classifies maps with lfc fibers.
\end{defn}

\begin{defn}
An \emph{ordinary Dirichlet functor} is a Dirichlet functor $D$ for which,
equivalently:
\begin{enumerate}
\item $D$ is cartesian over $\fst : \type{lfc}_{\ast} \to \type{lfc}$,
\item The fibers of $D!_0$ are locally finite and connected.
\end{enumerate}
\end{defn}


\section{Appendix}
\begin{lem}\label{lem:slice.cone.equiv}
There is an equivalence $S^{\triangleleft} \simeq \Hom_{\Ha}(!_0, S)$ over
$\down$, natural in $S$.
\end{lem}
\begin{proof}
\jaz{TODO}
\end{proof}

\begin{lem}\label{lem:cone.initial.equivalence}
Let $\Ca$ be an $\infty$-category with an initial object. For any simplicial set
$J$, let $\Ca^{J^{\triangleleft}}_{\ast}$ be the $\infty$-category of
  $J^{\triangleleft}$-shaped diagrams whose apex is at the initial object; that
  is, the following pullback:
  \[
    \begin{tikzcd}
      \Ca^{J} \arrow[drr, bend left] \arrow[ddr, bend right] \arrow[dr,
      dotted, "\sim"] & & \\
      & \Ca^{J^{\triangleleft}}_{\ast} \arrow[d] \arrow[r] \arrow[dr, phantom,
      "\ulcorner" very near start] & \Ca^{J^{\triangleleft}} \arrow[d,
      "\term{ev}_{\term{apex}}"] \\
      & \ast \arrow[r, "0"'] & \Ca
    \end{tikzcd}
  \]
The the dotted extension map  $\epsilon : \Ca^{J} \to \Ca^{J^{\triangleleft}}_{\ast}$ is an equivalence.
\end{lem}
\begin{proof}
\jaz{To Type}
\end{proof}

\begin{lem}\label{lem:kan.extending.cone.adds.initial}
If $\Ca$ is an $\infty$-category with an initial object, then the restriction
$\Ca^{J^{\triangleleft}} \to \Ca^{J}$ has a left adjoint right inverse $\epsilon
: \Ca^J \to \Ca^{J^{\triangleleft}}$ extending a diagram by adjoining the
initial object. 
\end{lem}
\begin{proof}
\jaz{TODO}
\end{proof}


\begin{lem}\label{lem:extending.diagram.with.initial.object.has.same.colimit}
If $\Ca$ is an $\infty$-category with an initial object $0$, and $d : D \to
\Ca^J$ a family of diagrams with colimit $\colim d : D \to \Ca$, then the family
$D \xto{d} \Ca^J \xto{\epsilon} \Ca^{J^{\triangleleft}}$ has colimit $\colim d$.
\end{lem}
\begin{proof}
For $\colim d : D \to \Ca$ to be the colimit of $d : D \to \Ca^J$ means that

\[
\begin{tikzcd}
\arrow[dr, phantom, "\scriptstyle\Uparrow\lambda" pos=.85] & B \arrow[d, "f"] & & X \arrow[dr, phantom, "\scriptstyle\Uparrow\chi"]\arrow[r, "b"] \arrow[d, "c"'] & B \arrow[d, "f"] & \arrow[d, phantom, "="] &  X \arrow[dr, phantom, "\scriptstyle\exists!\Uparrow\zeta" pos=.1,  "\scriptstyle\Uparrow\lambda" pos=.85 ]\arrow[r, "b"] \arrow[d, "c"'] & B \arrow[d, "f"]  \\ C \arrow[r, "g"'] \arrow[ur, "\ell"] & A & & C \arrow[r, "g"'] & A &~ & C \arrow[ur, "\ell" description] \arrow[r, "g"'] & A
\end{tikzcd}
\]


\end{proof}


\begin{lem}\label{lem:every.type.weakly.contractible.colimit}
Every homotopy type $S$ is the colimit of the weakly contractible diagram
$S^{\triangleleft} \to \ast^{\triangleleft} \xto{!_0} \Ha$.
\end{lem}
\begin{proof}
We note that $!_0$ is the extension of the diagram $\ast : \ast \to \Ha$ by an
initial object. Therefore, by Lemma
\ref{lem:extending.diagram.with.initial.object.has.same.colimit}, this diagram
has the same colimit as $S \to \ast \xto{\ast} \Ha$. The colimit of this diagram
is $S$, because $\Ha$ is generated by the point $\ast$ --- see (cite). 
\end{proof}

\begin{lem}\label{lem:hom.type.arrow}
  The following square is a pullback: 
  \[
% https://tikzcd.yichuanshen.de/#N4Igdg9gJgpgziAXAbVABwnAlgFyxMJZABgBpiBdUkANwEMAbAVxiRAB12AJCAWwH1gnLnQB6Q9lAgB3MAF85ACgC0pZQEoQc0uky58hFGQCMVWoxZthfQcLpLOU3qQAEjvpu27seAkWPkZvTMrIgc3DYSIg7sAMbQrpzxUJ46IBg+Bv6kptTBlmHWAlH2iu7ObnHQnmYwUADm8ESgAGYATnxIZCA4EEjGXiDtnYgBPX2IAMyDw7xIAEzUvUjTabNdSxPzchRyQA
\begin{tikzcd}
{\Hom_{\Ha^{\down}}(-,-)} \arrow[d] \arrow[r] \arrow[dr, phantom, "\ulcorner"
very near start] & {\Hom_{\Ha}(\cod, \cod)} \arrow[d] \\
{\Hom_{\Ha}(\dom, \dom)} \arrow[r]            & {\Hom_{\Ha}(\dom, \cod)}          
\end{tikzcd}
  \]
\end{lem}
\begin{proof}
We will give a presentation of this square in terms of cotensors by simplicial
sets. The square of simiplicial sets will be a pushout, and accordinly this
square will be a pullback.

We begin by noting that $\Hom_{\Ha}(\cod, \cod) \simeq \Ha^{\sqcup}$ where
\[
% https://tikzcd.yichuanshen.de/#N4Igdg9gJgpgziAXAbVABwnAlgFyxMJZABgBpiBdUkANwEMAbAVxiRAB1206AnPRzgBEYDHHQB6ARhABfUuky58hFGUlVajFm07c+WAe2GiJkgASc8AW3gWjIsVNnyQGbHgJFJ5DfWatEDnsTJzkFd2UvUnVqP21AzjgARwBjJjRZDRgoAHN4IlAAMx4IKyQyEBwIJAAmajgACyxCnHLqMSwGNgaICABrEGoGLDAAkDgIYahBkAaYOmnEMCYGBjCQYtLa9urEAGZ1zbLECqqkbxAGOgAjEQAFRQ8VEB4sHIbW2K0xgApOLCgpDMxAAlM4iiVjhczvtDpC2pVdgcKDIgA
\begin{tikzcd}
\partial\Delta^1 \arrow[r, hook] \arrow[d, "{(\id, 1)}"'] \arrow[rd, phantom, "\ulcorner" very near end] & \Delta^1 \arrow[d] \\
\partial\Delta^1 \times \Delta^1 \arrow[r]                           & \sqcup            
\end{tikzcd}
\]
and therefore we have a pullback square:
\[
\begin{tikzcd}
\Ha^{\sqcup} \arrow[r] \arrow[d] \arrow[rd, phantom, "\ulcorner" very near
start] & \Ha^{\Delta^1} \arrow[d] \\
\Ha^{\partial\Delta^1 \times \Delta^1} \arrow[r] & \Ha^{\partial \Delta^1}
\end{tikzcd}
\]
But the cospan in this pullback square is equivalently
\[
\begin{tikzcd}
\Ha^{\sqcup} \arrow[r] \arrow[d] \arrow[rd, phantom, "\ulcorner" very near
start] & \Ha^{\down} \arrow[d] \\
\Ha^{\down}\times \Ha^{\down} \arrow[r] & \Ha \times \Ha
\end{tikzcd}
\]
giving us the equivalent $\Hom_{\Ha}(\cod, \cod) \simeq \Ha^{\sqcup}$. We get
similar equivalences $\Hom_{\Ha}(\dom, \dom) \simeq \Ha^{\sqcap}$ and
$\Hom_{\Ha}(\dom, \cod) \simeq \Ha^{\term{N}}$ where
\[
% https://tikzcd.yichuanshen.de/#N4Igdg9gJgpgziAXAbVABwnAlgFyxMJZABgBpiBdUkANwEMAbAVxiRAB1206AnPRzgBEYDHHQB6ARhABfUuky58hFGUlVajFm07c+WAe2GiJkgASc8AW3gWjIsVNnyQGbHgJFJ5DfWatEDnsTJzkFd2UvUnVqP21AzjgARwBjJjRZDRgoAHN4IlAAMx4IKyQyEBwIJAAmajgACyxCnHLqMSwGNgaICABrEGoGLDAAkDgIYahBkAaYOmnEMCYGBjCQYtLa9urEAGZ1zbLECqqkbxAGOgAjEQAFRQ8VEB4sHIbW2K0xgApOLCgpDMxAAlM4iiVjhczvtDpC2pVdgcKDIgA
\begin{tikzcd}
\partial\Delta^1 \arrow[r, hook, "i"] \arrow[d, "{(\id, i)}"'] \arrow[rd, phantom, "\ulcorner" very near end] & \Delta^1 \arrow[d] \\
\partial\Delta^1 \times \Delta^1 \arrow[r]                           & \term{N}            
\end{tikzcd}
\]

Now, let $\underline{\term{N}}$ be the pushout
\[
\begin{tikzcd}
\Delta^0 \arrow[r, hook, "1"] \arrow[d, "2"'] \arrow[rd, phantom, "\ulcorner" very near end] & \Delta^1 \arrow[d] \\
\Delta^2 \arrow[r]                           & \underline{\term{N}}            
\end{tikzcd}
\]
We note that $\sqcup$ may also be constructed as the similar pushout
\[
\begin{tikzcd}
\Delta^0 \arrow[r, hook, "1"] \arrow[d, "2"'] \arrow[rd, phantom, "\ulcorner" very near end] & \Delta^1 \arrow[d] \\
\Lambda^2_{1} \arrow[r]                           & \underline{\term{N}}            
\end{tikzcd}
\]
and therefore $\sqcup \hookrightarrow \underline{\term{N}}$ is a trivial
cofibration in the Joyal model structure as it is a pushout of trivial
cofibrations. This tells us that the restriction map $\Ha^{\underline{\term{N}}}
\xto{\sim} \Ha^{\sqcup}$ is an equivalence. 
We have a similar story for $\Ha^{\overline{\term{N}}} \xto{\sim} \Ha^{\sqcap}$.
Thus, all things told, we may replace the square in the lemma statement by the
equivalent square:
\[
\begin{tikzcd}
\Ha^{\Delta^1 \times \Delta^1} \arrow[r] \arrow[d]  & \Ha^{\underline{\term{N}}} \arrow[d] \\
\Ha^{\overline{\term{N}}} \arrow[r] & \Ha^{\term{N}} 
\end{tikzcd}
\]
Finally, we note that the following square is a pushout of simplicial sets, so
that we get that our desired square is a pullback:
\[
\begin{tikzcd}
\term{N} \arrow[r, hook, "1"] \arrow[d, "2"'] \arrow[rd, phantom, "\ulcorner" very near end] & \underline{\term{N}} \arrow[d] \\
\overline{\term{N}} \arrow[r]                           & \Delta^1 \times \Delta^1            
\end{tikzcd}
\]
\end{proof}




%% Notes from Emily:
%
% Weber: 2-toposes, for learning about multivariable adjunctions with modules
%
% Arin Mazel-gee: wfs in oo-cats (check thesis or write to him)
%
% lifting orthogonal factorization from base to total space of cartesian
% fibration
%
% Emily says: the join of A and B is the Pi along * * >--> Delta^1 of A + B (up
% to iso)! So this gives a mapping in property to S^\triangleleft and so
% we can compare this mapping in property to the pullback property of Hom_H(!_0,
% S)
% 
% 5.3: the derivator proof should work. or: if C has an initial object, then C^J does
%
